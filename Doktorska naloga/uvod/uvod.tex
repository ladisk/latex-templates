% !TeX spellcheck = si_SI
\chapter{Uvod}\label{cha:uvod}

Uvodno poglavje naj vsebuje splošni opis oziroma razlago obravnavane tematike. Predstavite izhodišča doktorskega dela, problematiko in njegov pomen. V uvodu ne predstavljajte rezultatov in sklepov.

\section{Navodilo za uporabo predloge za doktorsko delo}\label{sec:uporaba_predloge}
V tej predlogi je \textbf{v nevsebinskem delu} doktorskega dela (do strani xxii) z [oglatimi oklepaji] označen tisti del besedila, ki ga mora študent oz. študentka spremeniti, da bo ustrezal njegovim oz. njenim podatkom ter podatkom o njegovem oz. njenem doktorskem delu. Ker uporabljate \LaTeX~se Kazalo vsebine, Kazalo slik ter Kazalo preglednic obnovijo ob vsakem prevajanju.\\

V tej predlogi so \textbf{v vsebinskem delu} doktorskega dela navodila in primeri za oblikovanje doktorskega dela. Celotno besedilo (ostanejo glavni naslovi: \nameref{cha:uvod}, \nameref{cha:teoreticne_osnove} itn.) mora študent oz. študenta nadomestiti z besedilom, ki vsebinsko ustreza njegovemu oz. njenemu doktorskemu delu.

\subsection{Uporaba prednastavljenih slogov v predlogi}\label{sec:prednastavitve}
Za pisanje doktorskega dela uporabljajte to predlogo, v kateri so že \textbf{prednastavljeni slogi} za poenotenje končne oblike doktorskih del na FS.\\

Kot je razvidno iz predloge, za naslove uporabljate naslednje ukaze:
\begin{itemize}
\item \verb|\chapter| za glavne naslove (npr. \ref{cha:uvod} \nameref{cha:uvod}),
\item \verb|\section| za naslov 2. ravni (npr. \ref{sec:enacbe} \nameref{sec:enacbe}),
\item \verb|\subsection| za naslov 3. ravni (npr. \ref{sec:vzorci_lit} \nameref{sec:vzorci_lit}),
\item \verb|\subsubsection| za naslov 4. ravni (npr. \ref{sec:poglavje_3} \nameref{sec:poglavje_3}),
\item \verb|\begin{itemize}\item\end{itemize}| za navajanje alinej.
\end{itemize}

Za \textbf{naslove slik in preglednic} (in tudi številčenje enačb) uporabljajte možnost samodejnega številčenja, in sicer pod sliko ali nad preglednico. Naslov slike ali preglednice definirate z ukazom \verb|\caption{<>}|, oznako slike, enačbe in preglednice definirate z \verb|\label{<>}|, na njih se pa sklicujete z \verb|\ref{<>}|. Tak način omogoča enostavno samodejno številčenje slik in preglednic (npr. da ni potrebno ročno popravljati številčenja, če v besedilu vrinete novo sliko oz. preglednico) ter tudi enostavno izdelavo seznama slik oz. seznama preglednic. \LaTeX ovi makri skrbijo, da se vsa številčena polja ob prevajanju samodejno posodobijo, vključno s seznami v nevsebinskem delu.\\

Za vstavljanje enačbe, kot je npr. enačba (\ref{eqn:e}) v poglavju \ref{sec:enacbe} \nameref{sec:enacbe}, uporabite okolje \verb|\begin{equation}<>\end{equation}|.

\subsection{Sklicevanje na dele besedila}\label{sec:sklici}

Sklicevanje na sliko, preglednico, enačbo ali del besedila je v \LaTeX u precej enostavno in enoznačno. Bodite pozorni, da pri sklicu na sliko oz. preglednico uporabite malo začetnico npr. \verb|slika \ref{<>}|. Pri sklicu na enačbo prav tako uporabite malo začetnico in postavite številko enačbe v oklepaje npr. \verb|enačba (\ref{<>})|.








