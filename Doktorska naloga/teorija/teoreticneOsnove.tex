% !TeX spellcheck = si_SI
\chapter{Teoretične osnove}\label{cha:teoreticne_osnove}


V primeru, da je obravnavana tematika tesno povezana s širšim znanstvenim področjem oz. poznavanjem določenih pojavov, procesov, postopkov, preračunov itn., ki jih doktorsko delo sicer ne obravnava neposredno, so pa pomembni za njegovo razumevanje, se lahko za poglavjem \ref{cha:uvod} \nameref{cha:uvod} umesti še poglavje \nameref{cha:teoreticne_osnove}. V tem poglavju predstavimo omenjeno širšo tematiko, predpostavke, razpon obstoječih metod, modelov, postopkov, njihove omejitve in relevantnost za doktorsko delo. Takega poglavja se poslužujemo tudi npr., če je obseg \nameref{cha:pregled_stanja}, na katerem temelji doktorsko delo, še relativno majhen in je potrebna nekoliko širša razlaga za lažje razumevanje obravnavane tematike. Naslov poglavja lahko prilagodimo obravnavani tematiki.\\

To poglavje ni obvezni del doktorskega dela in je lahko delno ali v celoti vključeno tudi v poglavju \nameref{cha:pregled_stanja}.

